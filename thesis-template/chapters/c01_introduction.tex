Blockchain which was first presented in 2008 by Satoshi Nakamoto \cite{sn}, is an emerging technology with a breakthrough potential. Since 2016, Blockchain have been listed in the Gartner’s Technology Trends reports \cite{c-1}, and it is expected to revolutionize the IT, business, and society around the world \cite{c-Deloi}.

In this chapter, I will present the overview of how industrial and academical field think of blockchain technology, how it caught our attention, what is the purpose of this master thesis and how to conduct the ideas. This chapter will give readers rather a complete plan of the thesis .\\

\section{Background}
Though the blockchain technology has been highlighted as the most innovative technologies in the following few years, yet still some skeptics think of it as a hype. However, according to a new market research report,the blockchain market size is expected to grow from USD 411.5 Million in 2017 to USD 7,683.7 Million 2022. Blockchain technology is on the fast lane toward widespread adoption, including in financial services industries, asset management, authentication, IoT, medical areas.

Why this disruptive technology becomes many forward-thinking enterprises' center of the attention? Nowadays, many companies compete one another beyond marketing strategies, research and development. Actually the competition has already extended to technical innovation, especially digital transformation. This transformation is beginning with finance and supply chain, two corporate and agency pillars ready to embrace all things digital. 

Digital business is shaping our business model in new era. It is changing how businesses communicate, transact and interact with customers, suppliers and clients. In a world already operating 24/7, how can businesses do not just keep pace – but run ahead of the competition today? How do they adapt to, and exceed, ever-evolving global customer demands and expectations? One of their ability to do so is largely dependent on their supply chain.

Thus higher requirements for supply chain comparing to traditional system will be the following:
\begin{itemize}
	\item \textbf{Higher visibility and transparency}\\
	New generation of supply chain should provide visibility
	into all aspects of the supply network, making
	it possible to dynamically track material flows,	
	synchronize schedules, balance supply with
	demand, and drive efficiencies. It also enables
	rapid, no-latency responses to changing network
	conditions and unforeseen disruptions.
	\item \textbf{Security capabilities} \\Reduce the concern of data encryption and confidentiality
	\item \textbf{Reliable vendor-client relationships}\\ Trading runs globally, your upstream suppliers may be located in another end of the earth, and very likely the clients are from several other countries. How could we enhance the truthful partnership among the parties for making smoother and more reliable transactions.
	
	\item \textbf{Robust and resilient} \\
	Being able to recover fast from cyber attack and failure. Partly small error and failure should not have great impact on the whole operation.
	\item \textbf{Better scalability}\\ This will help the system sustain the performance though when a large number of members joined the system.
\end{itemize} 
Facing the challenges and high demand, some companies proactively seek new solutions based on blockchain.  	


\subsection{Benefits of Blockchain-based System}
The blockchain system obtains a number of advantages over traditional centralized ledgers, databases. And that's why, the blockchain technology received massive positive feedback and draw great attention from all fields. The following items are the most general features of distributed ledger:
\begin{itemize}
	\item Decentralization
	
	- This is the essential part of blockchain technology. It means that there is no need for a trusted third party or intermediary to validate transactions.
	
	\item Greater transparency
	
	- Every transaction is recorded on the ledger, which can be seen by any party.
	
	\item Automation and programmability
	
	- All the business can be programed and preset in the system, this is also known as "smart contract". Blockchain can help reduce the tedious steps for setting up business
	 
	\item Cost efficiency
	
	- No third party or clearing houses are required in the blockchain, this can massively eliminate overhead costs in the form of fees.
	
	\item Immutability
	
	- Once the data has been written into the blockchain, it is extremely difficult to change it back. It is not truly immutable but, due to the fact that changing data is extremely difficult and almost impossible.
	
	\item Resilience from failure
	
	- Blockchain system is based on thousands of nodes in a peer-to-peer network, and the data is replicated and updated on each and every node. When one node fails or is attacked, it can easily restore the database from other nodes.
	
\end{itemize}


\subsection{Purpose of this Thesis}
Given sufficient theoretical evidence and positive prospect on the application of blockchain-based supply chain, we also care about the realizability, performance, and the usability. The best way to testify the value of blockchain-related technologies in Supply Chain field is to design and implement a prototype. Via testing and observation, we can get a basic conclusion, which is good reference for companies willing to dabble in blockchain field but with out too much resource and time to try out.

Since the whole Supply Chain Management System would be overly complex to be selected as the test sample, we noticed that as essential part of the logistics, Turnover Box, its circulatory may heavily impact the delivery of merchandise. 

In this thesis, I will design, implement and evaluate a blockchain-based turnover box system, and analyze the performance.